\section{Realted Work}
\label{sec:rel}

\stitle{Subgraph Enumeration.}  Besides the sequential algorithms we have mentioned in Section \ref{sec:intro}, GraphQL \cite{graph-ql} and SPath \cite{spath} focus on reducing the candidates of query vertices by exploiting neighborhood-based filtering. \turboiso \cite{turbo-iso} and the boost technique in \cite{subgraph-boost} propose to merge vertices with same labels and the same neighbours in $q$ and $G$ respectively to reduce the matching complexity. \cite{comparison} provides an in-depth comparison of above mentioned subgraph isomorphism algorithms. A more recent work in \cite{bi-fei} uses a data structure called \textit{compact path index} (CPI) to store the potential embeddings of a spanning tree of the query graph to improve both time and space efficiency. Algorithms of subgraph enumeration mainly focuses on answering a single query, \cite{multi-query} studied the problem of \textit{multiple query optimization} (MQO) for subgraph enumeration. The details of distributed subgraph enumeration algorithms can be found in Section \ref{sec:intro}.

\stitle{Subgraph Containment Search.} Let $\mathcal{D}=\{g_1,g_2,\dots,g_N\}$ be a graph database that has $N$ graphs, the problem of subgraph containment search over a graph database is to identify if the graphs in $\mathcal{D}$ contain the given query graph $q$. To speed up the search, many graph-feature based approaches have been proposed, performing graph indexing and adopting a filter-and-verification framework. As a result of such approach, false positives are removed by a pruning strategy before subgraph isomorphism algorithm is performed on each of the remaining candidates to obtain the final results. Existing works includes \textit{frequent subgraph mining based approaches} (e.g., gIndex \cite{gindex}, Tree+$\Delta$ \cite{tree+}, and FG-Index \cite{fg-index}) and \textit{exhaustive enumeration based approaches} (e.g., gCode \cite{gcode}, CT-Index \cite{ct-index}, GraphGrep \cite{graph-grep}, GraphGrepSX \cite{graph-grep-sx}, Closure-tree \cite{closure-tree}, and Grapes \cite{grapes}). In approximate graph containment search, TALE \cite{tale} was proposed.